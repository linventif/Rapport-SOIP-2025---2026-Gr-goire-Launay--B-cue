
% Master 1 MIAGE
% UE professionnalisation
% RAPPEL DES ATTENDUS
% 2 ateliers obligatoires au cours des deux semestres.
%  CV et lettre de motivation (1h30)
%  Construire son projet de stage (1h30)
% Puis 3 ateliers au choix (4h30) au cours des deux semestres dans ceux proposés au SOIP.
% COMPOSITION DU DOSSIER
% version imprimée à rendre à l’accueil du SOIP avant le vendredi 3 avril 12h30 2026
%  Page de garde (Nom, prénom, formation)
%  CV
%  Liste des ateliers suivis
%  Réflexion sur le projet de stage (ou emploi/alternance) en suivant la trame proposée
%  Synthèse personnelle de 10/15 ligne pour chaque atelier que vous avez choisi
% La rédaction de ce dossier a pour objectif de vous apprendre à donner du sens à vos études et vous
% permettre de vous poser les bonnes questions pour construire votre projet professionnel.
% Un projet est en constante évolution, il se construit tout au long de vos études, au travers de vos
% expériences et des opportunités
% Ce document est plutôt orienté vers le projet de stage mais vous pouvez en toute autonomie adapter ces
% questions à des projets d’alternance ou d’emploi.
% Dans le cadre de l’UE professionnelle, voici la trame des questions auxquelles vous devez répondre.
% 2
% TITRE : PROJET DE VOTRE CHOIX (STAGE/ALTERNANCE/EMPLOI)
% LES OBJECTIFS
% Pourquoi faire un stage ? Qu’est-ce que j’en attends ?
% Quel intérêt pour « l’entreprise » de prendre un stagiaire ?
% MISSIONS ET DES ACTIVITES QUI M’INTERESSENT
% Comment construire mon projet de stage ?
%  Quel lien le stage peut-il avoir avec mes études ?
%  Quel lien a-t-il avec mon projet professionnel ?
%  Quelles sont les connaissances et les compétences que je souhaite mobiliser ?
%  Quelles sont les missions, les activités, les tâches que j’aimerais assurer ?
%  Qu’est-ce que je voudrais éviter ? Et pourquoi ? ; …
% Comment définir la ou les missions du stage ? Quelles sont les informations nécessaires ?
%  Quels sont les secteurs d’activité en lien avec ma formation ?
%  Quelles entreprises pourraient être intéressées par mes compétences ?
%  Quels types de stages sont proposés ?
%  Dans quelle direction, quel service ?
%  Quelles sont les fonctions de l’entreprise correspondant à ma formation ? ; …
% LES MODALITES DU STAGE
% Quelles sont les modalités d’organisation du stage ?
%  Période ? Durée envisagée (ou obligatoire) ?
%  Quelles procédures administratives et réglementations (convention de stage, assurance) ?
% Quelles sont les modalités d’organisation dans l’entreprise visée ?
%  L’entreprise accueille-t-elle des stagiaires aux dates et durées qui me conviennent ?
%  Comment l'encadrement des stagiaires est-il assuré ?
%  Gratification des stagiaires ?
% REPERER LES ATOUTS
% Quels sont mes points forts, mes atouts pour le stage ?
% Identifier mes connaissances, mes compétences, mes atouts personnels à travers mes expériences
% Quels sont les critères pertinents pour le choix d'un lieu de stage ?
%  Secteur d’activité en expansion ?
%  Type d’entreprise (grande entreprise, PME/PMI, start-up, cabinet, collectivité, etc.) ?
%  Atouts de l’entreprise visée : taille, place sur le marché, dynamisme, etc. ?
% REPERER LES CONTRAINTES
% Quelles sont mes contraintes personnelles ?
%  Familiales
%  Géographiques (moyen de transport et de logement)
%  Financières
%  Activités extra universitaires (sport en compétition, jobs, engagements divers) ; …
% Quelles sont les contraintes du stage dans l’entreprise visée ?
%  Dates possibles ou périodes d’accueil des stagiaires
%  Localisation du stage : niveau local, régional, national, international
%  Coûts liés au transport, au logement, à la nourriture, etc.
%  Conditions de travail : extérieur, horaires, etc.
%  Responsabilités particulières : projet à mener dans un temps contraint, gestion de budget,
% encadrement d’équipe; …
% SYNTHESE DE MON PROJET
% Je rédige un projet de 10/15 lignes qui reprend en synthèse le projet sur lequel j’ai travaillé.
% 3
% TITRE : ATELIER AU CHOIX 1
% Je rédige une synthèse de 10/15 lignes qui reprend ce que j’ai retenu de cet atelier
% TITRE : ATELIER AU CHOIX 2
% Je rédige une synthèse de 10/15 lignes qui reprend ce que j’ai retenu de cet atelier
% TITRE : ATELIER AU CHOIX 3
% Je rédige une synthèse de 10/15 lignes qui reprend ce que j’ai retenu de cet atelier

\documentclass{article}
\usepackage[T1]{fontenc}
\usepackage[utf8]{inputenc}
\usepackage[french]{babel}
\usepackage{graphicx} % Required for inserting images
\usepackage{geometry}
\usepackage{fancyhdr} % For headers and footers
\usepackage{pdfpages} % For including PDF files
\geometry{a4paper, margin=2.5cm}

% Configure footer
\pagestyle{fancy}
\fancyhf{} % Clear all header and footer fields
\fancyfoot[C]{Université Toulouse Capitole 2025-2026 -- Grégoire Launay--Bécue -- M1 MIAGE}
\renewcommand{\headrulewidth}{0pt}
\renewcommand{\footrulewidth}{0.4pt}

\title{Rapport SOIP 2025 - 2026\\Master 1 MIAGE\\UE Professionnalisation}
\author{}
\date{}

\begin{document}

% PAGE DE GARDE
\begin{titlepage}
    \centering

    % Logo UT Capitole (à ajouter)
    % Placez le logo dans le même dossier que main.tex et décommentez la ligne suivante
    % \includegraphics[width=0.4\textwidth]{logo_ut_capitole.png}

    \vspace{2cm}

    {\LARGE\bfseries Rapport SOIP\par}
    \vspace{0.5cm}
    {\Large Master 1 MIAGE\par}
    \vspace{0.3cm}
    {\large UE Professionnalisation\par}

    \vfill

    {\large Année universitaire 2025-2026\par}

    \vspace{2cm}

    {\Large\bfseries Grégoire Launay--Bécue\par}

    \vspace{1cm}
\end{titlepage}

\newpage

% SOMMAIRE
\tableofcontents
\thispagestyle{fancy}

\newpage

% CV
\section*{Curriculum Vitae}
\addcontentsline{toc}{section}{Curriculum Vitae}

% Option 1: Inclure un PDF (recommandé)
% Placez votre CV au format PDF dans le même dossier que main.tex et décommentez la ligne suivante :
% \includepdf[pages=-]{cv.pdf}

% Option 2: Inclure une image (si le CV est en JPG/PNG)
% \begin{center}
%     \includegraphics[width=\textwidth,height=0.9\textheight,keepaspectratio]{cv.jpg}
% \end{center}

% Pour l'instant, espace réservé pour le CV
\vspace{1cm}
\begin{center}
    \textit{[Insérer votre CV ici en utilisant l'une des options ci-dessus]}
\end{center}

\newpage

% LISTE DES ATELIERS SUIVIS
\section*{Liste des ateliers suivis}
\addcontentsline{toc}{section}{Liste des ateliers suivis}

\subsection*{Ateliers obligatoires}
\begin{itemize}
    \item Rédiger son CV et sa lettre de motivation (1h30) - Date : vendredi 26 septembre 2025 (14h00-15h30)
    \item Construire son projet de stage (1h30) - Date : mercredi 19 novembre 2025 (17h00-18h30)
\end{itemize}

\subsection*{Ateliers au choix (3 ateliers - 4h30)}
\begin{enumerate}
    \item Pimp my LinkedIn - Date : vendredi 26 septembre 2025 (12h30-14h00)
    \item Postuler pour une mobilité internationale : CV et lettre - Date : jeudi 13 novembre 2025 (12h30-14h00)
    \item Développer son réseau - Date : jeudi 13 novembre 2025 (14h00-15h30)
\end{enumerate}

\newpage

% PROJET DE STAGE
\section{Projet de stage}

\subsection{Les objectifs}

\subsubsection{Pourquoi faire un stage ? Qu'est-ce que j'en attends ?}

[Votre réponse ici]

\subsubsection{Quel intérêt pour l'entreprise de prendre un stagiaire ?}

[Votre réponse ici]

\subsection{Missions et activités qui m'intéressent}

\subsubsection{Comment construire mon projet de stage ?}

\paragraph{Quel lien le stage peut-il avoir avec mes études ?}
[Votre réponse ici]

\paragraph{Quel lien a-t-il avec mon projet professionnel ?}
[Votre réponse ici]

\paragraph{Quelles sont les connaissances et les compétences que je souhaite mobiliser ?}
[Votre réponse ici]

\paragraph{Quelles sont les missions, les activités, les tâches que j'aimerais assurer ?}
[Votre réponse ici]

\paragraph{Qu'est-ce que je voudrais éviter ? Et pourquoi ?}
[Votre réponse ici]

\subsubsection{Comment définir la ou les missions du stage ? Quelles sont les informations nécessaires ?}

\paragraph{Quels sont les secteurs d'activité en lien avec ma formation ?}
[Votre réponse ici]

\paragraph{Quelles entreprises pourraient être intéressées par mes compétences ?}
[Votre réponse ici]

\paragraph{Quels types de stages sont proposés ?}
[Votre réponse ici]

\paragraph{Dans quelle direction, quel service ?}
[Votre réponse ici]

\paragraph{Quelles sont les fonctions de l'entreprise correspondant à ma formation ?}
[Votre réponse ici]

\subsection{Les modalités du stage}

\subsubsection{Quelles sont les modalités d'organisation du stage ?}

\paragraph{Période ? Durée envisagée (ou obligatoire) ?}
[Votre réponse ici]

\paragraph{Quelles procédures administratives et réglementations (convention de stage, assurance) ?}
[Votre réponse ici]

\subsubsection{Quelles sont les modalités d'organisation dans l'entreprise visée ?}

\paragraph{L'entreprise accueille-t-elle des stagiaires aux dates et durées qui me conviennent ?}
[Votre réponse ici]

\paragraph{Comment l'encadrement des stagiaires est-il assuré ?}
[Votre réponse ici]

\paragraph{Gratification des stagiaires ?}
[Votre réponse ici]

\subsection{Repérer les atouts}

\subsubsection{Quels sont mes points forts, mes atouts pour le stage ?}
[Votre réponse ici]

\subsubsection{Identifier mes connaissances, mes compétences, mes atouts personnels à travers mes expériences}
[Votre réponse ici]

\subsubsection{Quels sont les critères pertinents pour le choix d'un lieu de stage ?}

\paragraph{Secteur d'activité en expansion ?}
[Votre réponse ici]

\paragraph{Type d'entreprise (grande entreprise, PME/PMI, start-up, cabinet, collectivité, etc.) ?}
[Votre réponse ici]

\paragraph{Atouts de l'entreprise visée : taille, place sur le marché, dynamisme, etc. ?}
[Votre réponse ici]

\subsection{Repérer les contraintes}

\subsubsection{Quelles sont mes contraintes personnelles ?}

\paragraph{Familiales}
[Votre réponse ici]

\paragraph{Géographiques (moyen de transport et de logement)}
[Votre réponse ici]

\paragraph{Financières}
[Votre réponse ici]

\paragraph{Activités extra-universitaires (sport en compétition, jobs, engagements divers)}
[Votre réponse ici]

\subsubsection{Quelles sont les contraintes du stage dans l'entreprise visée ?}

\paragraph{Dates possibles ou périodes d'accueil des stagiaires}
[Votre réponse ici]

\paragraph{Localisation du stage : niveau local, régional, national, international}
[Votre réponse ici]

\paragraph{Coûts liés au transport, au logement, à la nourriture, etc.}
[Votre réponse ici]

\paragraph{Conditions de travail : extérieur, horaires, etc.}
[Votre réponse ici]

\paragraph{Responsabilités particulières : projet à mener dans un temps contraint, gestion de budget, encadrement d'équipe}
[Votre réponse ici]

\subsection{Synthèse de mon projet}

[Rédigez une synthèse de 10-15 lignes qui reprend l'ensemble de votre projet]

\newpage

% ATELIERS AU CHOIX
\section{Synthèses des ateliers au choix}

\subsection{Atelier au choix 1 : Pimp my LinkedIn}

\textbf{Description :} Optimisez votre profil LinkedIn pour vous démarquer. Découvrez comment valoriser vos expériences, élargir votre réseau et utiliser efficacement LinkedIn pour booster vos opportunités professionnelles.

\vspace{0.5cm}

\textbf{Synthèse personnelle (10-15 lignes) :}

[Rédigez une synthèse de 10-15 lignes qui reprend ce que vous avez retenu de cet atelier]

\subsection{Atelier au choix 2 : Postuler pour une mobilité internationale : CV et lettre}

\textbf{Description :} Atelier consacré à la préparation des candidatures pour des opportunités à l'international.

\vspace{0.5cm}

\textbf{Synthèse personnelle (10-15 lignes) :}

[Rédigez une synthèse de 10-15 lignes qui reprend ce que vous avez retenu de cet atelier]

\subsection{Atelier au choix 3 : Développer son réseau}

\textbf{Description :} Comprendre l'intérêt du réseau professionnel et identifier son propre réseau. Présentation pratique des réseaux sociaux professionnels et notamment de la plateforme alumni.

\vspace{0.5cm}

\textbf{Synthèse personnelle (10-15 lignes) :}

[Rédigez une synthèse de 10-15 lignes qui reprend ce que vous avez retenu de cet atelier]

\end{document}
