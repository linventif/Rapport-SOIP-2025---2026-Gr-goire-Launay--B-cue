
% Master 1 MIAGE
% UE professionnalisation
% RAPPEL DES ATTENDUS
% 2 ateliers obligatoires au cours des deux semestres.
%  CV et lettre de motivation (1h30)
%  Construire son projet de stage (1h30)
% Puis 3 ateliers au choix (4h30) au cours des deux semestres dans ceux proposés au SOIP.
% COMPOSITION DU DOSSIER
% version imprimée à rendre à l’accueil du SOIP avant le vendredi 3 avril 12h30 2026
%  Page de garde (Nom, prénom, formation)
%  CV
%  Liste des ateliers suivis
%  Réflexion sur le projet de stage (ou emploi/alternance) en suivant la trame proposée
%  Synthèse personnelle de 10/15 ligne pour chaque atelier que vous avez choisi
% La rédaction de ce dossier a pour objectif de vous apprendre à donner du sens à vos études et vous
% permettre de vous poser les bonnes questions pour construire votre projet professionnel.
% Un projet est en constante évolution, il se construit tout au long de vos études, au travers de vos
% expériences et des opportunités
% Ce document est plutôt orienté vers le projet de stage mais vous pouvez en toute autonomie adapter ces
% questions à des projets d’alternance ou d’emploi.
% Dans le cadre de l’UE professionnelle, voici la trame des questions auxquelles vous devez répondre.
% 2
% TITRE : PROJET DE VOTRE CHOIX (STAGE/ALTERNANCE/EMPLOI)
% LES OBJECTIFS
% Pourquoi faire un stage ? Qu’est-ce que j’en attends ?
% Quel intérêt pour « l’entreprise » de prendre un stagiaire ?
% MISSIONS ET DES ACTIVITES QUI M’INTERESSENT
% Comment construire mon projet de stage ?
%  Quel lien le stage peut-il avoir avec mes études ?
%  Quel lien a-t-il avec mon projet professionnel ?
%  Quelles sont les connaissances et les compétences que je souhaite mobiliser ?
%  Quelles sont les missions, les activités, les tâches que j’aimerais assurer ?
%  Qu’est-ce que je voudrais éviter ? Et pourquoi ? ; …
% Comment définir la ou les missions du stage ? Quelles sont les informations nécessaires ?
%  Quels sont les secteurs d’activité en lien avec ma formation ?
%  Quelles entreprises pourraient être intéressées par mes compétences ?
%  Quels types de stages sont proposés ?
%  Dans quelle direction, quel service ?
%  Quelles sont les fonctions de l’entreprise correspondant à ma formation ? ; …
% LES MODALITES DU STAGE
% Quelles sont les modalités d’organisation du stage ?
%  Période ? Durée envisagée (ou obligatoire) ?
%  Quelles procédures administratives et réglementations (convention de stage, assurance) ?
% Quelles sont les modalités d’organisation dans l’entreprise visée ?
%  L’entreprise accueille-t-elle des stagiaires aux dates et durées qui me conviennent ?
%  Comment l'encadrement des stagiaires est-il assuré ?
%  Gratification des stagiaires ?
% REPERER LES ATOUTS
% Quels sont mes points forts, mes atouts pour le stage ?
% Identifier mes connaissances, mes compétences, mes atouts personnels à travers mes expériences
% Quels sont les critères pertinents pour le choix d'un lieu de stage ?
%  Secteur d’activité en expansion ?
%  Type d’entreprise (grande entreprise, PME/PMI, start-up, cabinet, collectivité, etc.) ?
%  Atouts de l’entreprise visée : taille, place sur le marché, dynamisme, etc. ?
% REPERER LES CONTRAINTES
% Quelles sont mes contraintes personnelles ?
%  Familiales
%  Géographiques (moyen de transport et de logement)
%  Financières
%  Activités extra universitaires (sport en compétition, jobs, engagements divers) ; …
% Quelles sont les contraintes du stage dans l’entreprise visée ?
%  Dates possibles ou périodes d’accueil des stagiaires
%  Localisation du stage : niveau local, régional, national, international
%  Coûts liés au transport, au logement, à la nourriture, etc.
%  Conditions de travail : extérieur, horaires, etc.
%  Responsabilités particulières : projet à mener dans un temps contraint, gestion de budget,
% encadrement d’équipe; …
% SYNTHESE DE MON PROJET
% Je rédige un projet de 10/15 lignes qui reprend en synthèse le projet sur lequel j’ai travaillé.
% 3
% TITRE : ATELIER AU CHOIX 1
% Je rédige une synthèse de 10/15 lignes qui reprend ce que j’ai retenu de cet atelier
% TITRE : ATELIER AU CHOIX 2
% Je rédige une synthèse de 10/15 lignes qui reprend ce que j’ai retenu de cet atelier
% TITRE : ATELIER AU CHOIX 3
% Je rédige une synthèse de 10/15 lignes qui reprend ce que j’ai retenu de cet atelier

\documentclass{article}
\usepackage[T1]{fontenc}
\usepackage[utf8]{inputenc}
\usepackage[french]{babel}
\usepackage{graphicx} % Required for inserting images
\usepackage{geometry}
\usepackage{fancyhdr} % For headers and footers
\usepackage{pdfpages} % For including PDF files
\usepackage{textcomp} % For euro symbol
\geometry{a4paper, margin=2.5cm}

% Configure footer
\pagestyle{fancy}
\fancyhf{} % Clear all header and footer fields
\fancyfoot[C]{Université Toulouse Capitole 2025-2026 -- Grégoire Launay--Bécue -- M1 MIAGE}
\renewcommand{\headrulewidth}{0pt}
\renewcommand{\footrulewidth}{0.4pt}

\title{Rapport SOIP 2025 - 2026\\Master 1 MIAGE\\UE Professionnalisation}
\author{}
\date{}

\begin{document}

% PAGE DE GARDE
\begin{titlepage}
    \centering

    \vspace{2cm}

    {\LARGE\bfseries Rapport SOIP\par}
    \vspace{0.5cm}
    {\Large Master 1 MIAGE\par}
    \vspace{0.3cm}
    {\large UE Professionnalisation\par}

    \vfill
    % Logo UT Capitole
    \includegraphics[width=0.4\textwidth]{./ut-capitole.jpg}
    \vfill

    {\large Année universitaire 2025-2026\par}

    \vspace{1cm}

    {\Large\bfseries Grégoire Launay--Bécue\par}

    \vspace{1cm}
\end{titlepage}\newpage

% SOMMAIRE
\tableofcontents
\thispagestyle{fancy}

\newpage

% CV
\section*{Curriculum Vitae}
\addcontentsline{toc}{section}{Curriculum Vitae}

% Option 1: Inclure un PDF (recommandé)
% Placez votre CV au format PDF dans le même dossier que main.tex et décommentez la ligne suivante :
% \includepdf[pages=-]{cv.pdf}

% Option 2: Inclure une image (si le CV est en JPG/PNG)
% \begin{center}
%     \includegraphics[width=\textwidth,height=0.9\textheight,keepaspectratio]{cv.jpg}
% \end{center}

% Pour l'instant, espace réservé pour le CV
\vspace{1cm}

\includepdf[pages=1]{./cv_gregoire_launay-becue.pdf}

\newpage

% LISTE DES ATELIERS SUIVIS
\section*{Liste des ateliers suivis}
\addcontentsline{toc}{section}{Liste des ateliers suivis}

\subsection*{Ateliers obligatoires}
\begin{itemize}
    \item Rédiger son CV et sa lettre de motivation (1h30) - Date : vendredi 26 septembre 2025 (14h00-15h30)
    \item Construire son projet de stage (1h30) - Date : mercredi 19 novembre 2025 (17h00-18h30)
\end{itemize}

\subsection*{Ateliers au choix (3 ateliers - 4h30)}
\begin{enumerate}
    \item Pimp my LinkedIn - Date : vendredi 26 septembre 2025 (12h30-14h00)
    \item Postuler pour une mobilité internationale : CV et lettre - Date : jeudi 13 novembre 2025 (12h30-14h00)
    \item Développer son réseau - Date : jeudi 13 novembre 2025 (14h00-15h30)
\end{enumerate}

\newpage

% PROJET DE STAGE
\section{Projet de stage}

\subsection{Les objectifs}

\subsubsection{Pourquoi faire un stage ? Qu'est-ce que j'en attends ?}

Ce stage est l'occasion d'appliquer mes apprentissages MIAGE à un contexte industriel exigeant.
J'en attends une montée en compétence concrète sur l'IA appliquée à la téléphonie et au cloud.
Je souhaite aussi développer mon réseau professionnel auprès d'équipes expertes en télécom et produits SaaS.
Je travaille justement sur un projet personnel de transcription audio vers texte et de classification, ce qui renforce mon intérêt pour cette offre.

\subsubsection{Quel intérêt pour l'entreprise de prendre un stagiaire ?}

OVHcloud bénéficie d'un regard neuf pour identifier des usages IA différenciants sur la téléphonie.
Je peux prototyper rapidement, documenter et tester des idées qui alimentent la roadmap produit.
Le stage offre également une ressource dédiée pour explorer et dé-risquer des POC sans bloquer les équipes cœur.

\subsection{Missions et activités qui m'intéressent}

\subsubsection{Comment construire mon projet de stage ?}

\paragraph{Quel lien le stage peut-il avoir avec mes études ?}

La MIAGE m'apporte une double compétence management/SI qui colle à la téléphonie cloud.
Je peux articuler architecture VoIP, développement logiciel et compréhension des besoins métier.

\paragraph{Quel lien a-t-il avec mon projet professionnel ?}
Je vise un poste d'ingénieur logiciel orienté IA/VoIP dans le cloud.
Ce stage Téléphonie IA chez OVHcloud est aligné sur ce projet et me donne une première immersion produit.

\paragraph{Quelles sont les connaissances et les compétences que je souhaite mobiliser ?}
Développement Python (et initiation Go), modèles d'IA (NLP, reconnaissance vocale), APIs REST/GraphQL.
Pratiques Git/CI, conteneurisation, monitoring, et sensibilité aux contraintes de sécurité/télécom.

\paragraph{Quelles sont les missions, les activités, les tâches que j'aimerais assurer ?}
Prototyper des services IA pour la téléphonie : transcription, résumé d'appels, agent d'accueil intelligent.
Construire des scripts d'intégration CRM (pré-remplissage, catégorisation) et des démonstrateurs.
Tester/déployer les POC et préparer des supports de présentation pour les décideurs.

\paragraph{Qu'est-ce que je voudrais éviter ? Et pourquoi ?}
Des tâches de pure maintenance sans lien avec l'IA ou la voix.
Travailler isolé sans retour utilisateur ou sans objectifs mesurables, car cela limiterait l'apprentissage.

\subsubsection{Comment définir la ou les missions du stage ? Quelles sont les informations nécessaires ?}

\paragraph{Quels sont les secteurs d'activité en lien avec ma formation ?}
Cloud public, télécom/VoIP, CPaaS, centres de contact, SaaS B2B, et solutions IA appliquées.

\paragraph{Quelles entreprises pourraient être intéressées par mes compétences ?}
OVHcloud (téléphonie et cloud), opérateurs alternatifs, éditeurs de softswitch (Asterisk/Freeswitch), intégrateurs CRM/téléphonie, start-up de voicebots.

\paragraph{Quels types de stages sont proposés ?}
R\&D IA, prototypage produit, intégration API VoIP/SMS, industrialisation de POC, analyse de performance vocale.

\paragraph{Dans quelle direction, quel service ?}
Equipe Télécom/OneTeam d'OVHcloud, en lien avec le Product Manager et les experts IA.
Interface possible avec les équipes R\&D et les équipes commerciales pour les démonstrations.

\paragraph{Quelles sont les fonctions de l'entreprise correspondant à ma formation ?}
Développeur backend/Go, ingénieur IA/data, intégration produit, tests/QA orientés charge, support technique sur des prototypes.

\subsection{Les modalités du stage}

\subsubsection{Quelles sont les modalités d'organisation du stage ?}

\paragraph{Période ? Durée envisagée (ou obligatoire) ?}

La période est imposée par le M1 MIAGE : d'avril à août 2026 (4 à 5 mois).
L'offre OVHcloud précise une arrivée le 20/04/2026 et une fin au 21/08/2026 (4 mois).

\paragraph{Quelles procédures administratives et réglementations (convention de stage, assurance) ?}
Convention de stage UT Capitole, attestation d'assurance RC, accord des conditions de sécurité/systèmes d'information, éventuellement NDA, et respect du RGPD pour les données vocales.

\subsubsection{Synthèse des conditions proposées par OVHcloud}

\begin{itemize}
    \item Poste : Software Engineer (stage téléphonie/IA) à Roubaix, 35h/semaine.
    \item Contrat : convention de stage de 4 mois (20/04/2026 au 21/08/2026).
    \item Absences : une semaine de congés payés autorisée.
    \item Gratification : 1200 \texteuro{}/mois.
    \item Avantages : cantine (0,80 \texteuro{}/repas), remboursement 50\% de l'abonnement transport, accès aux avantages du CSE.
    \item Contact RH : Magali de Labareyre (Tech Talent Acquisition Manager).
\end{itemize}

\subsubsection{Quelles sont les modalités d'organisation dans l'entreprise visée ?}

\paragraph{L'entreprise accueille-t-elle des stagiaires aux dates et durées qui me conviennent ?}
Oui, le stage Téléphonie IA est prévu sur site à Roubaix avec télétravail hybride et correspond à la période visée.

\paragraph{Comment l'encadrement des stagiaires est-il assuré ?}
Processus décrit : échange avec la TA (Magali), suivi par le manager (Merwan) et un pair de l'équipe.
Rituels d'équipe et revues de code permettront d'avoir du feedback régulier.

\paragraph{Gratification des stagiaires ?}
Gratification conforme au minimum légal (>= 4,35 EUR/h en 2026) et avantages OVHcloud (titre/restauration, transport, accès plateforme de formation).

\subsection{Repérer les atouts}

\subsubsection{Quels sont mes points forts, mes atouts pour le stage ?}
Autonomie, goût pour l'innovation, pratique de Python et bases en Go, curiosité pour la téléphonie et l'IA.
Capacité à documenter et vulgariser pour les démonstrations produit.

\subsubsection{Identifier mes connaissances, mes compétences, mes atouts personnels à travers mes expériences}
Projets MIAGE en data/IA et en développement web, pratique Git/CI, notions de conteneurisation.
Expériences associatives qui renforcent l'organisation et la prise de parole.

\subsubsection{Quels sont les critères pertinents pour le choix d'un lieu de stage ?}

\paragraph{Secteur d'activité en expansion ?}
Cloud souverain, téléphonie IP et IA conversationnelle sont en forte croissance.

\paragraph{Type d'entreprise (grande entreprise, PME/PMI, start-up, cabinet, collectivité, etc.) ?}
Grande entreprise tech (comme OVHcloud) ou scale-up B2B pour bénéficier d'une culture produit structurée.

\paragraph{Atouts de l'entreprise visée : taille, place sur le marché, dynamisme, etc. ?}
OVHcloud : acteur majeur du cloud libre, portefeuille VoIP/SMS, investissements IA, environnement multiculturel et équipements adaptés.

\subsection{Repérer les contraintes}

\subsubsection{Quelles sont mes contraintes personnelles ?}

\paragraph{Familiales}
Prévoir des retours réguliers dans ma région d'origine ; besoin d'un cadre compatible avec ces déplacements.

\paragraph{Géographiques (moyen de transport et de logement)}
Stage basé à Roubaix : recherche de logement en métropole lilloise et budget transport (train/métro).

\paragraph{Financières}
Budget étudiant : la gratification doit couvrir une partie du logement et des transports.

\paragraph{Activités extra-universitaires (sport en compétition, jobs, engagements divers)}
Quelques engagements associatifs : besoin d'une organisation hebdomadaire claire pour maintenir l'équilibre.

\subsubsection{Quelles sont les contraintes du stage dans l'entreprise visée ?}

\paragraph{Dates possibles ou périodes d'accueil des stagiaires}
Aligner le démarrage avec la fenêtre universitaire (avril) et la durée souhaitée (4-6 mois).

\paragraph{Localisation du stage : niveau local, régional, national, international}
Localisation principale : Roubaix (Hauts-de-France), avec possibilité ponctuelle de télétravail.

\paragraph{Coûts liés au transport, au logement, à la nourriture, etc.}
Coûts de logement métropole lilloise, transports quotidiens, et restauration ; prévoir aides (bourses/logement étudiant).

\paragraph{Conditions de travail : extérieur, horaires, etc.}
Bureaux équipés, travail essentiellement sur site/ordinateur, horaires de bureau avec flexibilité liée au modèle hybride.

\paragraph{Responsabilités particulières : projet à mener dans un temps contraint, gestion de budget, encadrement d'équipe}
Livrer des POC dans des délais courts, préparer des démonstrations pour les décideurs et mesurer l'impact utilisateur.

\subsection{Synthèse de mon projet}

Je candidate et suis retenu pour un stage Téléphonie IA chez OVHcloud à Roubaix.
Objectif : connecter mes compétences MIAGE à des cas d'usage IA/VoIP (transcription, résumé, agent d'accueil).
Le stage associe prototypage, tests et démonstrations auprès du Product Manager et de l'équipe Télécom.
Il s'inscrit dans mon projet professionnel d'ingénieur logiciel orienté IA/VoIP dans le cloud.
Je mobiliserai Python (et Go), mes notions d'IA, l'intégration d'API VoIP/SMS et les bonnes pratiques Git/CI.
Les avantages OVHcloud (hybride, formation, environnement multiculturel) soutiennent mon apprentissage.
Contraintes identifiées : logement/transport à Roubaix et synchronisation avec le calendrier universitaire.
Atouts personnels : autonomie, curiosité télécom, capacité à vulgariser et à livrer des POC démontrables.
Le stage me permettra d'explorer les produits VoIP/SMS d'OVHcloud, de proposer des améliorations et de mesurer leur impact.
Objectif final : livrer au moins un prototype démontrable et documenté pour les décideurs.
Je partagerai un retour d'expérience pour capitaliser sur les apprentissages au sein de la promotion MIAGE.

\newpage

% ATELIERS AU CHOIX
\section{Synthèses des ateliers au choix}

\subsection{Atelier au choix 1 : Pimp my LinkedIn}

\textbf{Description :} Optimisez votre profil LinkedIn pour vous démarquer. Découvrez comment valoriser vos expériences, élargir votre réseau et utiliser efficacement LinkedIn pour booster vos opportunités professionnelles.

\vspace{0.5cm}
\textbf{Synthèse personnelle (10-15 lignes) :}

Pendant cet atelier, nous avons étudié les bases de LinkedIn : création de compte et structuration du profil.
J'ai appris à soigner la photo, le titre professionnel et le résumé pour capter l'attention des recruteurs.
La rédaction des expériences et la mise en avant des compétences et des réalisations a été détaillée.
Nous avons vu l'importance des recommandations et des compétences validées pour renforcer la crédibilité.
L'atelier a expliqué comment identifier et élargir son réseau en ciblant des contacts pertinents.
Des bonnes pratiques de prise de contact et d'interaction (messages, commentaires, publications) ont été présentées.
Nous avons abordé les techniques de recherche d'offres et l'utilisation des filtres pour optimiser les candidatures.
La personnalisation des messages de candidature et le suivi des candidatures ont été fortement recommandés.
Un volet était dédié à la veille d'entreprise et à l'exploitation des pages entreprises pour préparer les entretiens.
Les limites et les atouts de LinkedIn Premium ont été brièvement évoqués et situés selon les besoins.
Enfin, j'en retiens des actions concrètes : améliorer mon titre, enrichir mon résumé, solliciter des recommandations et planifier des prises de contact régulières.


\subsection{Atelier au choix 2 : Postuler pour une mobilité internationale : CV et lettre}

\textbf{Description :} Atelier consacré à la préparation des candidatures pour des opportunités à l'international.

\vspace{0.5cm}

\textbf{Synthèse personnelle (10-15 lignes) :}

Pendant cet atelier, nous avons exploré les spécificités des candidatures pour une mobilité internationale.
Nous avons vu l'importance d'adapter le CV et surtout d'écrire une lettre de motivation au Président de l'université.
L'atelier a mis en avant la nécessité de bien comprendre les attentes des comités de sélection.
Nous avons appris à valoriser les expériences interculturelles et les compétences linguistiques.
Nous avons aussi eu un échange avec un ancien étudiant ayant effectué une mobilité internationale, ce qui a permis d'avoir un retour d'expérience concret.
Les plannings et les deadlines pour les candidatures ont été clairement expliqués.
L'atelier a souligné l'importance de la préparation aux entretiens, souvent réalisés en anglais.
Nous avons discuté des erreurs courantes à éviter dans les candidatures internationales.
Enfin, j'ai compris que la mobilité internationale est une opportunité précieuse pour enrichir son parcours académique et professionnel, et qu'une candidature bien préparée est essentielle pour maximiser ses chances de succès.

\subsection{Atelier au choix 3 : Développer son réseau}

\textbf{Description :} Comprendre l'intérêt du réseau professionnel et identifier son propre réseau. Présentation pratique des réseaux sociaux professionnels et notamment de la plateforme alumni.

\vspace{0.5cm}

\textbf{Synthèse personnelle (10-15 lignes) :}

Nous avons passé en revue les différents types de réseaux : personnel, professionnel, académique, hobby, etc.
L'atelier a mis en avant l'importance de cultiver et d'entretenir son réseau régulièrement.
Nous avons vu comment identifier les contacts clés et les opportunités de réseautage.
En recherche d'emploi, le réseau peut ouvrir des portes et offrir des recommandations précieuses.
Nous avons exploré diverses stratégies pour approcher de nouveaux contacts de manière authentique.
Les réseaux sociaux professionnels, comme LinkedIn, sont des outils puissants pour élargir son réseau.
L'atelier a souligné l'importance de la réciprocité dans les relations professionnelles.
Nous avons appris à préparer des présentations efficaces pour les événements de réseautage.
L'atelier a également abordé les erreurs à éviter, comme le réseautage superficiel.
Nous avons discuté de l'importance de suivre et de remercier les contacts après les rencontres.
Enfin, j'ai compris que le développement du réseau est un processus continu qui nécessite du temps et de l'engagement.

\end{document}
